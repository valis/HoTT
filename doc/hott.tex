\documentclass{amsart}

\usepackage{amssymb}
\usepackage[all]{xy}
\usepackage{verbatim}
\usepackage{ifthen}
\usepackage{xargs}

\providecommand\WarningsAreErrors{false}
\ifthenelse{\equal{\WarningsAreErrors}{true}}{\renewcommand{\GenericWarning}[2]{\GenericError{#1}{#2}{}{This warning has been turned into a fatal error.}}}{}

\newcommand{\newref}[4][]{
\ifthenelse{\equal{#1}{}}{\newtheorem{h#2}[hthm]{#4}}{\newtheorem{h#2}{#4}[#1]}
\expandafter\newcommand\csname r#2\endcsname[1]{#3~\ref{#2:##1}}
\expandafter\newcommand\csname R#2\endcsname[1]{#4~\ref{#2:##1}}
\newenvironmentx{#2}[2][1=,2=]{
\ifthenelse{\equal{##2}{}}{\begin{h#2}}{\begin{h#2}[##2]}
\ifthenelse{\equal{##1}{}}{}{\label{#2:##1}}
}{\end{h#2}}
}

\newref[section]{thm}{theorem}{Theorem}
\newref{lem}{lemma}{Lemma}
\newref{prop}{proposition}{Proposition}
\newref{cor}{corollary}{Corollary}

\theoremstyle{definition}
\newref{defn}{definition}{Definition}
\newref{example}{example}{Example}

\theoremstyle{remark}
\newref{remark}{remark}{Remark}

\newcommand{\cat}[1]{\mathbf{#1}}
\newcommand{\C}{\cat{C}}
\newcommand{\bbG}{\mathbb{G}}
\newcommand{\Set}{\cat{Set}}
\newcommand{\PSh}[1]{\Set^{#1^{op}}}
\newcommand{\nats}{\mathbb{N}}
\newcommand{\glob}{\PSh{\bbG}}
\newcommand{\ocat}{\omega \cat{Cat}}
\newcommand{\D}[1]{\mathrm{D}^{#1}}
\newcommand{\Dn}{\D{n}}
\newcommand{\dD}[1]{\mathrm{\partial D}^{#1}}
\newcommand{\dDn}{\dD{n}}

\newcommand{\pb}[1][dr]{\save*!/#1-1.2pc/#1:(-1,1)@^{|-}\restore}
\newcommand{\po}[1][dr]{\save*!/#1+1.2pc/#1:(1,-1)@^{|-}\restore}

\numberwithin{figure}{section}

\begin{document}

\title{Homotopy Type Theory}

\author{Valery Isaev}

\begin{abstract}
In this paper, we give a description of an evaluation algorithm for homotopy type theory.
\end{abstract}

\maketitle

\section{Introduction}

\section{Cellular sets}

In this section, we will recall the definition of cellular sets.

\begin{defn}
A \emph{globular set} is a presheaf on the category $\bbG$, whose objects are natural numbers and morphisms are generated from
$\sigma_n,\tau_n : n \to n + 1$ subject to the equations
$\sigma_{n+1} \circ \sigma_n = \tau_{n+1} \circ \sigma_n$, $\sigma_{n+1} \circ \tau_n = \tau_{n+1} \circ \tau_n$.
A morphism of globular sets is a morphism of presheaves.
\end{defn}

Explicitly, a globular set is a sequence of sets $X_n$ together with maps $s_k,t_k : X_n \to X_k$ for each $k < n$ subject to equations.
Elements of $X_n$ are called $n$-cells. Given $n$-cell $x$, cells $s_k(x)$ and $t_k(x)$
are called the ($k$-dimensional) source and the target of $x$ respectively.
A pair of $n$-cells $x$ and $y$ are \emph{parallel} if $s_k(x) = s_k(y)$ and $t_k(x) = t_k(y)$ for each $k < n$.
The expression $x : a \to b$ stands for the following:
there is $n \in \nats$ such that $a$ and $b$ are (necessarily parallel) $n$-cells,
$x$ is an $(n+1)$-cell, $s_n(x) = a$, and $t_n(x) = b$.

For each $n \in \nats$, the globular set $\bbG(-,n)$ is denoted by $\Dn$.
The globular sets $\dDn$ together with the maps $\dDn \to \Dn$ are defined by induction on $n$.
$\dD{0}$ is the initial object, and $\dD{0} \to \D{0}$ is the unique morphism.
$\dD{n+1}$ is the pushout $\Dn \amalg_{\dDn} \Dn$, and $\dD{n+1} \to \D{n+1}$
is induced by the maps $(\sigma_n \circ -)$ and $(\tau_n \circ -)$.

\begin{defn}
A \emph{strict $\omega$-category} is a globular set equipped with the following operations.
\begin{itemize}
\item For each $n$-cell $x$, an $(n+1)$-cell $id_x : x \to x$ called the \emph{identity} on $x$.
\item For each $n,k \in \nats$ and for each pair of $n$-cells $x$,$y$ such that
$t_k(x) = s_k(y)$ (in this case we say $x$ and $y$ are \emph{$k$-composable}),
an $n$-cell $y *_k x$ called \emph{k-composite} of $x$ and $y$.
\end{itemize}
We often regard an $n$-cell $x$ as a cell of a dimension higher than $n$ iteratively applying $id$ to $x$.

The operations above must satisfy the following properties.
\begin{itemize}
\item If $x : a \to b$, $y : b \to c$ is a pair of $n$-composable $(n+1)$-cells, then $y *_n x : a \to c$.
\item If $k < n$ and $x : a \to b$, $y : c \to d$ is a pair of $k$-composable $(n+1)$-cells, then $y *_k x : c *_k a \to d *_k b$.
\item If $x,y,z$ are $n$-cells, then $(z *_k y) *_k x = z *_k (y *_k x)$, whenever it makes sense.
\item If $k < n$ and $x$ is an $n$-cell, then $t_k(x) *_k x = x = x *_k s_k(x)$.
\item If $k < n$ and $x$,$y$ is a pair of $k$-composable $n$-cells, then $id_y *_k id_x = id_{y *_k x}$.
\item If $x$,$y$,$z$,$u$ are $n$-cells and $m < k < n$, then $(x *_k y) *_m (z *_k u) = (x *_m z) *_k (y *_m u)$, whenever it makes sense.
\end{itemize}

An $\omega$-functor between a pair of $\omega$-categories is a morphism of
the underlying globular sets commuting with compositions and identities.
The category $\ocat$ has $\omega$-categories as objects and $\omega$-functors as morphisms.
\end{defn}

The forgetful functor $U : \ocat \to \glob$ has a left adjoint $F : \glob \to \ocat$.
Let us denote by $(T : \glob \to \glob, \eta_X : X \to T(X), \mu_X : T T(X) \to T(X))$ the monad $U \circ F$,
and by $\dDn_T$ and $\Dn_T$ the $\omega$-categories $T(\dDn)$ and $T(\Dn)$.

An \emph{$n$-dimensional pasting diagram} is defined by induction on $n$.
There is exactly one $0$-dimensional pasting diagram, which is denoted by $\bullet$.
An $(n+1)$-dimensional pasting diagram is a finite sequence of $n$-dimensional pasting diagrams.

With each $n$-dimensional pasting diagram $\pi$ we associate an $n$-dimensional globular set $\widehat{\pi}$.
If $\pi$ is the unique $0$-dimensional pasting diagram, then $\widehat{\pi}$ is the terminal $0$-dimensional globular set.
If $\pi = (\pi^1, \ldots \pi^k)$, then let $\widehat{\pi}_0 = \{0, 1, \ldots k\}$
and $\widehat{\pi}_{m+1} = \coprod \limits_{i = 1}^k \widehat{\pi^i}_m$.
For each $k > 0$, the source and the target are defined in the evident way,
and for each $x \in \widehat{\pi^i}$, we define $s_0(x) = i - 1$ and $t_0(x) = i$.

\begin{defn}
The category $\Theta$ is the full subcategory of $\ocat$ on strict $\omega$-categories of the form $T(\widehat{\pi})$ for each pasting diagram $\pi$.
A cellular set is a presheaf on $\Theta$, a morphism of cellular sets is a morphism of presheaves.
\end{defn}

\bibliographystyle{amsplain}
\bibliography{ref}

\end{document}
