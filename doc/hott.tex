\documentclass{amsart}

\usepackage{amssymb}
\usepackage[all]{xy}
\usepackage{verbatim}
\usepackage{ifthen}
\usepackage{xargs}
\usepackage{bussproofs}
\usepackage{stmaryrd}
\usepackage{mathrsfs}

\providecommand\WarningsAreErrors{false}
\ifthenelse{\equal{\WarningsAreErrors}{true}}{\renewcommand{\GenericWarning}[2]{\GenericError{#1}{#2}{}{This warning has been turned into a fatal error.}}}{}

\newcommand{\newref}[4][]{
\ifthenelse{\equal{#1}{}}{\newtheorem{h#2}[hthm]{#4}}{\newtheorem{h#2}{#4}[#1]}
\expandafter\newcommand\csname r#2\endcsname[1]{#3~\ref{#2:##1}}
\expandafter\newcommand\csname R#2\endcsname[1]{#4~\ref{#2:##1}}
\newenvironmentx{#2}[2][1=,2=]{
\ifthenelse{\equal{##2}{}}{\begin{h#2}}{\begin{h#2}[##2]}
\ifthenelse{\equal{##1}{}}{}{\label{#2:##1}}
}{\end{h#2}}
}

\newref[section]{thm}{theorem}{Theorem}
\newref{lem}{lemma}{Lemma}
\newref{prop}{proposition}{Proposition}
\newref{cor}{corollary}{Corollary}

\theoremstyle{definition}
\newref{defn}{definition}{Definition}
\newref{example}{example}{Example}

\theoremstyle{remark}
\newref{remark}{remark}{Remark}

\newcommand{\cat}[1]{\mathbf{#1}}
\newcommand{\D}{\mathsf{D}}
\newcommand{\bbG}{\mathbb{G}}
\newcommand{\Set}{\cat{Set}}
\newcommand{\PSh}[1]{\Set^{#1^{op}}}
\newcommand{\nats}{\mathbb{N}}
\newcommand{\Z}{\mathbb{Z}}
\newcommand{\glob}{\PSh{\bbG}}
\newcommand{\ocat}{\omega \cat{Cat}}
\newcommand{\Dn}[1][n]{\mathrm{D}^{#1}}
\newcommand{\dDn}[1][n]{\mathrm{\partial D}^{#1}}

\newcommand{\pb}[1][dr]{\save*!/#1-1.2pc/#1:(-1,1)@^{|-}\restore}
\newcommand{\po}[1][dr]{\save*!/#1+1.2pc/#1:(1,-1)@^{|-}\restore}

\numberwithin{table}{section}

\begin{document}

\title{Homotopy Type Theory}

\author{Valery Isaev}

\begin{abstract}
In this paper, we give a description of an evaluation algorithm for homotopy type theory.
\end{abstract}

\maketitle

\section{Introduction}

\section{Cubical sets}

In this section, we will recall the definition of cubical sets.
We use cubical sets with connections, but call them simply cubical sets.

Let $\Box$ be the following category. Its objects are natural numbers, and morphisms are generated from
\[ \partial^\alpha_i : n \to n + 1, \]
\[ \epsilon_i : n + 1 \to n, \]
\[ \Gamma^\alpha_j : n + 1 \to n, \]
for each $n \in \nats$, $0 \leq i \leq n$, $0 \leq j < n$, and $\alpha = \pm$.
These maps satisfy the following equations:
\[ \partial^\alpha_j \partial^\beta_i = \partial^\beta_i \partial^\alpha_{j-1} \text{ if $i < j$}, \]
\[ \epsilon_j \epsilon_i = \epsilon_i \epsilon_{j+1} \text{ if $i \leq j$}, \]
\[ \Gamma^\alpha_j \Gamma^\beta_i = \Gamma^\beta_i \Gamma^\alpha_{j+1} \text{ if $i < j$}, \]
\[ \Gamma^\alpha_i \Gamma^\alpha_i = \Gamma^\alpha_i \Gamma^\alpha_{i+1}, \]
\[ \epsilon_j \partial^\alpha_i = \left\{ \begin{array}{l l}
            \partial^\alpha_i \epsilon_{j-1} & \text{if $i < j$} \\
            id                               & \text{if $i = j$} \\
            \partial^\alpha_{i-1} \epsilon_j & \text{if $i > j$},
    \end{array} \right. \]
\[ \epsilon_j \Gamma^\alpha_i = \left\{ \begin{array}{l l}
            \Gamma^\alpha_i \epsilon_{j+1} & \text{if $i < j$} \\
            \epsilon_j \epsilon_j         & \text{if $i = j$} \\
            \Gamma^\alpha_{i-1} \epsilon_j & \text{if $i > j$},
    \end{array} \right. \]
\[ \Gamma^\beta_j \partial^\alpha_i = \left\{ \begin{array}{l l}
            \partial^\alpha_i \Gamma^\beta_{j-1} & \text{if $i < j$} \\
            id                                   & \text{if $j \leq i \leq j + 1$ and $\alpha = \beta$} \\
            \partial^\alpha_j \epsilon_j         & \text{if $j \leq i \leq j + 1$ and $\alpha = -\beta$} \\
            \partial^\alpha_{i-1} \Gamma^\beta_j & \text{if $i > j + 1$}.
    \end{array} \right. \]

A map in the category $\Box$ is \emph{injective} (resp., \emph{surjective}) if it is a composition of maps of the form $\partial^\alpha_i$ (resp., $\epsilon_i$ and $\Gamma^\alpha_i$).
A cubical set is a presheaf on the category $\Box$.

\section{Cubical syntax}

In this section, we will define cubical terms and inference rules for them.
For now, we define only $\Pi$ and $\Sigma$ types.
Later, we will extend the syntax by adding path types, (univalent) universes, and (higher) inductive types.

Let $Vars$ be a countably infinite set of variables.
Then we define the sets of raw terms and contexts as follows:
\[ Tm ::= \alpha^* x\ |\ \Pi (x : A) B\ |\ \lambda x. b\ |\ f\ a\ |\ \Sigma (x : A) B\ |\ a , b\ |\ proj_1\ p\ |\ proj_2\ p\ | \]
\[ Id\ A\ a\ a'\ |\ path\ a\ |\ J\ d\ p\ |\ coe\ A\ a\ |\ lift\ A\ a\ |\ comp\ A\ a\ a'\ a''\ |\ fill\ A\ a\ a'\ a'', \]
\[ Ctx ::= \varnothing\ |\ \Gamma, x :_k A, \]
where $x \in Vars$, $\Gamma \in Ctx$, $k \in \Z$, $a, a', a'', b, f, p, A, B \in Tm$, and $\alpha$ is a map in the category $\Box$.

Now, for each $\alpha : k \to n$ and term $t$, we define a term $\alpha^* t$.
Assuming that the sets of bounded and free variables of $t$ are disjoint, we can define $\alpha^* t$ as follows:
\begin{itemize}
\item[] $\alpha^* (\beta^* x) = \left\{\begin{array}{l l}
            (\beta \alpha)^* x  & \text{if $x \in FV(t)$} \\
            \beta^* x           & \text{otherwise,}
        \end{array} \right.$
\item[] $\alpha^* (\Pi (x : A) B) = \Pi (x : \alpha^* A) (\alpha^* B)$
\item[] $\alpha^* (\lambda x. b) = \lambda x. \alpha^* b$
\item[] $\alpha^* (f\ a) = \alpha^* f\ \alpha^* a$
\item[] $\alpha^* (\Sigma (x : A) B) = \Sigma (x : \alpha^* A) (\alpha^* B)$
\item[] $\alpha^* (a , b) = \alpha^* a , \alpha^* b$
\item[] $\alpha^* (proj_1\ p) = proj_1\ (\alpha^* p)$
\item[] $\alpha^* (proj_2\ p) = proj_2\ (\alpha^* p)$
\item[] $\alpha^* (Id\ A\ a\ a') = Id\ (\alpha_{+1}^* A)\ (\alpha^* a)\ (\alpha^* a')$
\item[] $\alpha^* (path\ a) = path\ (\alpha_{+1}^* a)$
\item[] $\alpha^* (J\ d\ p) = J\ (\alpha^* d)\ (\alpha^* p)$
\item[] $\alpha^* (coe\ A\ a) = coe\ (\alpha_{+1}^* A)\ (\alpha^* a)$
\item[] $\alpha^* (lift\ A\ a) = lift\ (\alpha_{+1}^* A)\ (\alpha^* a)$
\item[] $\alpha^* (comp\ A\ a\ a'\ a'') = comp\ (\alpha_{+1}^* A)\ (\alpha^* a)\ (\alpha^* a')\ (\alpha^* a'')$
\item[] $\alpha^* (fill\ A\ a\ a'\ a'') = fill\ (\alpha_{+1}^* A)\ (\alpha^* a)\ (\alpha^* a')\ (\alpha^* a'')$
\end{itemize}

\centerAlignProof

\begin{table}

Structural rules

\medskip
\begin{center}
\AxiomC{}
\UnaryInfC{$\varnothing \vdash$}
\DisplayProof
\quad
\AxiomC{$\Gamma \vdash A : Type_n$}
\RightLabel{, $x \notin \Gamma$}
\UnaryInfC{$\Gamma, x :_k A \vdash$}
\DisplayProof
\quad
\AxiomC{$\Gamma \vdash a : A$}
\AxiomC{$\Gamma \vdash B : Type_n$}
\RightLabel{, $A =_c B$}
\BinaryInfC{$\Gamma \vdash a : B$}
\DisplayProof
\end{center}

\medskip
\begin{center}
\AxiomC{$\Gamma \vdash A : Type_n$}
\RightLabel{, $x :_k A \in \Gamma, \alpha : m \to n$ is surjective with height at most $k$}
\UnaryInfC{$\Gamma \vdash \alpha^* x : \alpha^* A$}
\DisplayProof
\end{center}

\bigskip
Pi types

\medskip
\begin{center}
\AxiomC{$\Gamma \vdash A : Type_n$}
\AxiomC{$\Gamma, x :_0 A \vdash B : Type_n$}
\BinaryInfC{$\Gamma \vdash \Pi (x : A) B : Type_n$}
\DisplayProof
\end{center}

\medskip
\begin{center}
\AxiomC{$\Gamma, x :_0 A \vdash b : B$}
\AxiomC{$\Gamma \vdash A : Type_n$}
\BinaryInfC{$\Gamma \vdash \lambda x. b : \Pi (x : A) B$}
\DisplayProof
\quad
\AxiomC{$\Gamma \vdash f : \Pi (x : A) B$}
\AxiomC{$\Gamma \vdash a : A$}
\BinaryInfC{$\Gamma \vdash f\ a : B[x := a]$}
\DisplayProof
\end{center}

\bigskip
Sigma types

\medskip
\begin{center}
\AxiomC{$\Gamma \vdash A : Type_n$}
\AxiomC{$\Gamma, x :_0 A \vdash B : Type_n$}
\BinaryInfC{$\Gamma \vdash \Sigma (x : A) B : Type_n$}
\DisplayProof
\end{center}

\medskip
\begin{center}
\AxiomC{$\Gamma \vdash a : A$}
\AxiomC{$\Gamma \vdash b : B[x := a]$}
\AxiomC{$\Gamma \vdash \Sigma (x : A) B : Type_n$}
\TrinaryInfC{$\Gamma \vdash a, b : \Sigma (x : A) B$}
\DisplayProof
\end{center}

\medskip
\begin{center}
\AxiomC{$\Gamma \vdash p : \Sigma (x : A) B$}
\UnaryInfC{$\Gamma \vdash proj_1\ p : A$}
\DisplayProof
\quad
\AxiomC{$\Gamma \vdash p : \Sigma (x : A) B$}
\UnaryInfC{$\Gamma \vdash proj_2\ p : B[x := proj_1\ p]$}
\DisplayProof
\end{center}

\bigskip
Path types

\medskip
\begin{center}
\AxiomC{$\Gamma_{+1} \vdash A : Type_{n+1}$}
\AxiomC{$\Gamma \vdash a : {\partial^-_0}^* A$}
\AxiomC{$\Gamma \vdash a' : {\partial^+_0}^* A$}
\TrinaryInfC{$\Gamma \vdash Id\ A\ a\ a' : Type_n$}
\DisplayProof
\end{center}

\medskip
\begin{center}
\AxiomC{$\Gamma_{+1} \vdash a : A$}
\AxiomC{$\Gamma \vdash {\partial^-_0}^* a : {\partial^-_0}^* A$}
\AxiomC{$\Gamma \vdash {\partial^+_0}^* a : {\partial^+_0}^* A$}
\TrinaryInfC{$\Gamma \vdash path\ a : Id\ A\ ({\partial^-_0}^* a)\ ({\partial^+_0}^* a)$}
\DisplayProof
\end{center}

\medskip
\begin{center}
\Axiom$\fCenter \Gamma, a :_0 A, a' :_0 A, p :_0 Id\ (\epsilon_0^* A)\ a\ a' \vdash C : Type_n$
\noLine
\def\extraVskip{1pt}
\UnaryInf$\fCenter \Gamma \vdash d : \Pi (a : A) C[a' := a, p := idp\ a]$
\AxiomC{$\Gamma \vdash p : Id\ (\epsilon_0^* A)\ a\ a'$}
\def\extraVskip{2pt}
\BinaryInfC{$\Gamma \vdash J\ d\ p : C\ a\ a'\ p$}
\DisplayProof
\end{center}

\bigskip
Liftings

\medskip
\begin{center}
\AxiomC{$\Gamma_{+1} \vdash A : Type_1$}
\AxiomC{$\Gamma \vdash a : {\partial^-_0}^* A$}
\BinaryInfC{$\Gamma \vdash coe\ A\ a : {\partial^+_0}^* A$}
\DisplayProof
\quad
\AxiomC{$\Gamma \vdash A : Type_1$}
\AxiomC{$\Gamma \vdash a : {\partial^-_0}^* A$}
\BinaryInfC{$\Gamma \vdash lift\ A\ a : A$}
\DisplayProof
\end{center}

\medskip
\begin{center}
\AxiomC{$\Gamma_{+1} \vdash A : Type_{n+2}$}
\AxiomC{$\Gamma \vdash a : {\partial^-_0}^* A$}
\AxiomC{$\Gamma \vdash a' : {\partial^-_1}^* A$}
\AxiomC{$\Gamma \vdash a'' : {\partial^+_0}^* A$}
\QuaternaryInfC{$\Gamma \vdash comp\ A\ a\ a'\ a'' : {\partial^+_1}^* A$}
\DisplayProof
\end{center}

\medskip
\begin{center}
\AxiomC{$\Gamma \vdash A : Type_{n+2}$}
\AxiomC{$\Gamma \vdash a : {\partial^-_0}^* A$}
\AxiomC{$\Gamma \vdash a' : {\partial^-_1}^* A$}
\AxiomC{$\Gamma \vdash a'' : {\partial^+_0}^* A$}
\QuaternaryInfC{$\Gamma \vdash fill\ A\ a\ a'\ a'' : A$}
\DisplayProof
\end{center}

\bigskip
\caption{Inference rules.}
\label{table:inf-rules}
\end{table}

\bibliographystyle{amsplain}
\bibliography{ref}

\end{document}
